
\chapter*{Goals}\addcontentsline{toc}{chapter}{Goals}\markboth{Goals}{Goals}

The main goal of the thesis is to improve the implementation of structured orthogonal grids in the Template Numerical Library (TNL).
The list of issues that will be addressed as a part of the improvement in the grid implementation in the TNL is the following:

\begin{enumerate}
 \item TNL implements one-dimensional, two-dimensional and three-dimensional orthogonal grids, but each of them is implemented independently of the other.
       However, some grid properties don’t depend on the grid dimension.
       Therefore, the goal is to extract dimension-independent properties of the grid to reduce the repetitiveness in the codebase.
 \item The main operation performed on the grid is the parallel traversal of the grid entities.
       The current grid implementation in the TNL uses separate classes to define the traversal operation.
       Later, the TNL team introduces the optimal parallel for loop algorithm.
       The issue is that the traverse classes implement a similar algorithm for the traverse, which makes their implementation redundant.
       The goal is to update traverser classes to use the parallel algorithm.
 \item Various numerical solvers will use grid classes to perform computations.
       The usage of grid classes may add significant run-time overhead to the numerical solver.
       Therefore, the goal is to estimate the overhead added by the grid environment.
       To measure it, the numerical solver of the heat equation using the Finite Difference Method (FDM) based on the grid class will be implemented.
       It will be compared with the numerical solver, which solves the same equation, but uses only primitive functions provided by the TNL.
 \item Current grid implementation in the TNL is poorly tested.
       In addition to that, it must be noted that the TNL actively uses template metaprogramming.
       The compiler uses templates to generate specific data types at compile time, i.e. the source code will be generated and compiled by the end users of the library.
       As a result, different specializations of the templates must be tested because of the error possibility in the generated code.
       The goal is to cover the grid classes with the unit tests.
\end{enumerate}

The compulsory goal of the thesis is to learn CUDA programming.
It will be achieved by implementing CUDA kernels for the convolution operator.
There are several requirements for the implementation:

\begin{enumerate}
 \item The convolution operator must be optimal for the small kernels.
 \item The convolution operator must support one-, two- and three-dimensional kernels.
 \item The convolution operator must be generic, i.e. the user can define any convolution mask.
\end{enumerate}

To fulfill the requirements I will implement different versions of the convolution operator.
The goal is to learn CUDA primitives and to understand their effect on the computation efficiency, therefore not all variants of the convolution operator algorithms will be implemented.

Finally, the heat equation has a pseudo-analytical solution, which uses the convolution operator.
The goal is to use the implementation of the convolution operator and show the similarity between the numerical solution obtained with FDM and the pseudo-analytical solution to the heat equation.
