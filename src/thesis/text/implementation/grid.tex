\chapter{Grid} \label{sec:grid-implementation}

The TNL provides \texttt{Grid} and \texttt{GridEntity} class templates to work with structured orthogonal grids.
From the implementation perspective, the library aims to minimize the runtime overhead by implementing statically resolved types.
As a result, the compiler performs more compile-time optimizations.
In addition to that, both \texttt{Grid} and \texttt{GridEntity} aim to be lightweight and store only essential information.
The TNL supports one-, two- and three-dimensional grids.
The main operation performed on the grid is the parallel traverse of its entities.
The grid can traverse all, boundary and interior entities concurrently.
Additionally, the grid positions entities in space and allows them to operate with the entity neighbours.
The grid entity is not only the cell or the vertex of the grid, but the grid supports any subentity of the grid cell.
The TNL implements a projected coordinates system.
Each entity is indexed with the Cartesian coordinate, and then the coordinate is projected to its position in space.

Most of the values in the grid and grid entity are represented as vectors of the grid dimension length.
Hence, grid defines \texttt{Coordinate} and \texttt{Point} types, which represent the grid dimension-length vector of integer and real values accordingly.

\section{Grid entity}

The \texttt{GridEntity<Grid, EntityDimension>} class template represents the entity of a structured grid.
The \texttt{EntityDimension} template parameter specifies the dimension of the grid entity.
Possible dimensions of the grid entity are zero or the value lower or equal to the dimension of the grid.
The zero-dimensional grid entity represents a vertex, the one-dimensional grid entity represents an edge and so on.
The \texttt{GridEntity} is highly coupled with the grid it is defined in.
Therefore, the \texttt{Grid} template parameter references the type of the \texttt{Grid} class.
From the \texttt{Grid} class type the \texttt{GridEntity} gets the information about the dimension of the grid, index and real types.
The grid entity stores only essential information needed: the reference on the grid, the coordinates of the entity in the grid, its global index, orientation and basis (described in Section \ref{sec:basis}).

\subsection{Coordinates}

The grid indexes every entity using the Cartesian coordinate system.
In every cell, a coordinate is assigned to the vertex in the cell, which is the closest to the coordinate origin.
And the coordinate is the index of the vertex in the sequence of cells.
The example of the coordinates is shown in Figure \ref{fig:grid-entities}
The type of the coordinate is the \texttt{Coordinate}.
However, the coordinate is not capable of identifying all grid entities.
To show this let's consider 2-dimensional grid from Figure \ref{fig:grid-entities}.
\((0, 0)\) is the coordinate, which will uniquely identify both the vertex and the grid cell.
However, there are two types of edges which go from vertex \((0, 0)\), therefore the coordinate is not sufficient, which led to the introduction of the orientation.

The user can get coordinates from both the device and the host using the \texttt{getCoordinates()} method.
Additionally, the grid entity coordinates can be updated using the \texttt{setCoordinates()} method.

\subsection{Orientation}

The orientation represents the position of the entity relative to grid axes and is of integer type.
For example, let's consider a 3-dimensional grid.
There is only one way to position cell and vertex to grid axes, therefore they have the orientation \(0\).
There are 3 ways to position edges and faces to axes, therefore there are 3 possible orientations.

From the other point of view, the orientation characterizes the linear subspace to which the entity belongs.
To understand this more, let's show how the linear subspace is denoted.
The Cartesian coordinate system is a linear space, which is spanned by a set of linearly independent basis vectors.
All vectors in the set are unique, otherwise, they aren't linearly independent.
The selection of \(k\)-vectors from the basis forms a linearly independent set of vectors, which spans a linear subspace.
Using previous observations to generate all linear subspaces the enumeration of \((0, 1, \dots, n)\)-combinations is performed.
Each \(n\)-dimensional entity of the grid either belongs to the linear subspace of the corresponding dimension or the linear manifold collinear to it.
The orientation of \(n\)-dimensional entity identifies the combination index in the lexicographically sorted sequence of \(n\)-combinations from basis vectors.
All possible orientations are represented in Figure \ref{fig:grid-entities}.

The entity dimension and the orientation uniquely identify the \textit{entity type}.
Thus, both entity dimension and the orientation should be the template parameters in the grid entity type.
However, this affects the traverse operations.
The traverse operations must accept the lambda function by design, which will be called with the grid entity.
At the moment of writing, the CUDA programming model didn't support generic lambdas.
Therefore to traverse entities, the user must specify a lambda function for every orientation.
However, entities of different orientations are generally processed similarly.
Hence the grid entity class stores its orientation.

The user can access the orientation of the entity from both the device and the host by using the \texttt{getOrientation()} method of the \texttt{GridEntity}.
In addition to that, the grid entity orientation can be updated by the \texttt{setOrientation()} method.

\subsection{Basis} \label{sec:basis}

The basis is the \(n\)-dimensional boolean vector, which characterizes the grid entity type.
The value at index \(i\) in the vector is \(0\) if the entity spans along the \(i\)-th axis, otherwise, it is \(1\).
The examples of bases are listed in Figure \ref{fig:grid-entities}.
The type of basis is the \texttt{Coordinate}.

It must be mentioned, that the basis is superfluous to the entity dimension and the entity orientation.
Nonetheless, the basis has an important role in the traversal operations, verifying if the entity is a boundary and accessing the neighbour entities.
The main reason is the relationship between the entity type and the number of corresponding entities in the grid.
The grid stores its dimensions as the number of edges along its axes.
The basis possesses the following property: the grid dimensions are formed by entities of a specific dimension and the number of entities along grid axes is given by summing up the basis and grid dimensions.
In the following text, the term \textit{grid entities dimensions} will mean the dimension of the grid aligned with the basis.
For example, the dimensions of a 2-D grid on Figure \ref{fig:grid-entities} are \((5, 3)\).
The basis of the vertex is \((1, 1)\) and the number of vertices along axes is the following.

\begin{equation}
 (5, 3) + basis = (5, 3) + (1, 1) = (6, 4)
\end{equation}

\clearpage
\import{./text/imgs}{group}
\clearpage

In the implementation, the basis is generated using template metaprogramming.
The problem of generating bases is reduced to generating types, which represent the k-combinations.
In addition to that, the implementation of the basis generation is dimension independent.
The developer can generate \texttt{constexpr} basis value by using the static \texttt{getBasis<EntityOrientation>()} method in the \texttt{BasisGetter<EntityDimension>} class.
However, the compiler must know the orientation and dimension at the compilation phase.
Hence the grid additionally stores bases, which can be accessed dynamically by using \texttt{getBasis<EntityDimension>(orientation)} defined in the \texttt{Grid} class.

The user can access the basis of the entity from both the device and the host by using the \texttt{getBasis()} method of given \texttt{GridEntity}.
In addition to that, the grid entity basis can be updated by using the \texttt{setBasis()} method.

\subsection{Global index}

The global index is the grid entity integer index calculated from the entity coordinate and the orientation.
To calculate the index the grid entities counts are obtained from the basis and the dimensions of the grid following the method from Section \ref{sec:basis}.
Then the index is calculated from the coordinate of the entity the following way.

\begin{equation}
 \text{index} = c.x() + \text{dim}.x() * c.y() + \text{dim}.x() * \text{dim}.y() * c.z()
\end{equation}

The \(c\) is the coordinate of the grid entity and the \(dim\) is the grid entity dimensions.

For the entities with multiple orientations, this method will result in similar indices for different orientations.
However, entities of different types are processed independently, therefore it is enough to correctly align the entities' indices.
The index of the entity with the orientation \(k\) is adjusted by adding entity counts of lower orientations.
The example is on the Figure \ref{fig:memory-1-D}, where \((0)\), \((1)\), \((2)\) are orientations of the edges in the 3-dimensional grid.

\import{./text/imgs}{memory}

To access the global index of the entity the user can use \texttt{getIndex()} method.

\subsection{Boundary entities}

The grid entity is a boundary if all of its vertices are located on the surface of the grid.
The user can check if the entity is the boundary by using the \texttt{isBoundary()} method of the \texttt{GridEntity}.

\subsection{Neighbours}

In most cases, numerical methods perform operations on the entity and its neighbours.
The process of obtaining neighbour entities has two edge cases.
The first case is when the neighbour of the same dimension and the orientation to the parent entity are needed.
Then to get a neighbour it is enough to adjust the coordinate and update the global index because the entity of the same type has the same basis and orientation.
The second case is when the dimension and the orientation differ from the parent entity.
Then the coordinate and the global index are updated the same way, but in some cases, the neighbour entity doesn't support the parent orientation.
In these cases, the orientation of the neighbour entity is set to the maximally supported orientation and the basis is updated.

To get the neighbour entity the user can use any of the following methods:

\begin{itemize}
 \item{ \texttt{getNeighbourEntity<Dimension, Steps...>()} } returns the entity of the \texttt{Dimension} with the coordinates adjusted by \texttt{Steps...}
 \item{ \texttt{getNeighbourEntity<Dimension, Orientation, Steps...>()} } returns the entity of the \texttt{Dimension} and \texttt{Orientation} with the coordinates adjusted by \texttt{Steps...}
 \item{ \texttt{getNeighbourEntity<Dimension>(const Coordinate\& offset)} } returns the entity of the \texttt{Dimension} with the coordinates adjusted by \texttt{offset}.
 \item{ \texttt{getNeighbourEntity<Dimension, Orientation>(const Coordinate\& offset)} } returns the entity of the \texttt{Dimension} and \texttt{Orientation} with the coordinates adjusted by \texttt{offset}.
\end{itemize}

When the method is called with \texttt{Dimension} and \texttt{Orientation}, it will use the static basis value.
Otherwise, it will access the basis dynamically from the grid.
The same is true for the \texttt{Steps...} variadic parameter pack.

\subsection{Measure and center}

The grid entity implements \texttt{getMeasure()} and \texttt{getCenter()} methods to get the information about the entity position in the domain.
The measure represents the volume of the hypercube, which the entity represents, and the center represents the position of the grid entity center in the domain.

In general, both the measure and the center can be implemented using bases, because the basis represents the directions in which the entity spans.
However, the implementation will be less efficient, because the basis must be accessed dynamically.

\section{Grid}

The \texttt{Grid<Dimension, Real, Device, Index>} class template represents an orthogonal structured grid.
It inherits from \texttt{NDimGrid<Dimension, Real, Device, Index>}, which implements dimension independent operations on the grid.
The \texttt{Dimension} template parameter represents the dimension of the grid.
The \texttt{Device} represents the device, on which parallel traverse of grid entities will be performed.
The \texttt{Index} and \texttt{Real} represent index and real value correspondingly.
The grid stores information about entity counts, configurations of domain and bases of entities.

\subsection{Dimensions}

The dimension of the grid along the axis is the number of edges along the corresponding axis.
When the dimension of the grid is updated, the grid recalculates the count of every entity type in the grid and the proportions of the grid.
The entity counts are obtained the same way described in Section \ref{sec:basis}.

The user can use \texttt{getDimensions()} to get the dimensions of the grid.
The user can use \texttt{setDimensions()} to set the dimensions of the grid.

\subsection{Domain}

The implementation of the grid only supports cells of equal dimensions.
The user can specify the dimension of the cell by using \texttt{setSpaceSteps()} method.
The method accepts on the input the vector, where each real element at \(i\)-th index represent the dimension of the cell along the corresponding axis.
It is often needed to calculate the coefficients in the following form.

\begin{equation}
 h_x^i * h_y^j * h_z^k
\end{equation}

The \(h_{*}\) represents the space step along the \(*\)-th axis and \(i\), \(j\), \(k\) are integer powers of space steps.
For that reason, the grid calculates the coefficients for all \(i\), \(j\), \(k\) from range \([-2, 2]\).
The user can get the coefficient by using \texttt{getSpaceStepsProducts(Powers...)} and \texttt{getSpaceStepsProducts<Powers...>()} method.
By using the second method, the grid will calculate the index of the coefficient in coefficient lists at the compile time.

Additionally, the grid supports positioning in given space.
The origin represents the point, where the zero coordinate is located.
The type of origin is the \texttt{Point}.
The user can use \texttt{getOrigin()} to get the origin of the grid.
The user can use \texttt{setOrigin()} to set the origin of the grid.
The \texttt{setOrigin()} accepts on the input the vector of real values.

Finally, the user can specify the size of the grid and the origin using \texttt{setDomain()} method.
The grid will calculate the space steps based on their dimension and update the origin.

\subsection{Traverse}

The grid implements parallel traverse of all, boundary and interior entities.
To traverse grid entities the developer must use \texttt{forAll}, \texttt{forBoundary} and \texttt{forInterior} methods.
Because of the Cartesian coordinate system, the traversal of grid entities is the same as the iterating of the grid entities' coordinates.
The problem of iterating all coordinates for entities with a single orientation type is solved with several nested for loops.
It is because all entities are identified with the coordinates and all coordinates form rectangle.
For entities with multiple orientations grid entities dimensions differ for each orientation, but it can be resolved by processing each orientation independently.
In the following text, the \textit{start index} term will represent the initial values of nested for loops and the \textit {end index} will represent the end values of nested for loops.

The traverse of entities with multiple orientations is used in the traverse of all, boundary and interior entities.
Therefore static (compile-time) for loop \texttt{ForEachOrientation} was implemented using template metaprogramming.
The \texttt{exec()} function from \texttt{ForEachOrientation} accepts the lambda function on the input, which will be called with orientation and basis of entity.
The orientation has \texttt{std::integral\_constant} type, therefore it can be used in type definition.
Additionally, the \texttt{ForEachOrientation} can skip one orientation value.

Finally, the implementation of all entities of any dimension traversal is the following.

\vspace{8pt}

\begin{listing}[!h]
\caption{The implementation of all grid entities traversal}
\begin{minted}{c++}
Coordinate from = ..., to = ...;

auto exec = [&](const Index orientation, const Coordinate& basis) {
  ParallelFor<...>::exec(from, to + basis, ...);
};

ForEachOrientation<Index, EntityDimension, Dimension>::exec(exec);
\end{minted}
\end{listing}

\vspace{8pt}

The traversal of interior (non-boundary) entities is similar to the traversal of all entities, but the start and the end indices must be adjusted.
For entities with multiple orientations, the start index is adjusted by adding the basis vector.
By doing this the start index is moved along axes, such that the entity doesn't span along with them, therefore it isn't a boundary.
For entities with a single orientation, the start and end indices are adjusted depending on if the coordinate represents the grid vertex or cell.
The implementation of interior entity traversal is the following.

\begin{listing}[!h]
\caption{The implementation of interior grid entities traversal}
\begin{minted}{c++}
Coordinate from = ..., to = ...;

auto exec = [&](const Index orientation, const Coordinate& basis) {
  if (Dimension == EntityDimension) {
    ParallelFor<...>::exec(from + Coordinate(1), to - Coordinate(1), ...);
    return;
  }

  Templates::ParallelFor<...>::exec(from + basis, to, ...);
};

ForEachOrientation<Index, EntityDimension, Dimension>::exec(exec);
\end{minted}
\end{listing}
\newpage

\vspace{8pt}

The traversal of boundary entities is built on the fact, that the surface of the \(n\)-dimensional hypercube is formed by the \((n-1)\)-dimensional linear spaces.
There are exactly \(2n\) boundary surfaces of the hypercube.
Each surface is spanned by the set of \((n-1)\) independent vectors and the other basis vector is orthogonal to the surface.
Therefore, to iterate all surfaces it is enough to iterate all basis vectors or all orientations of grid surface dimension entity.
The traversal of the intersections formed by boundaries is an important implementation aspect because intersections belong to multiple surfaces.
Each entity must be traversed only once, therefore, entities, which belong to the intersections, must be addressed by the traversal algorithm.
The implementation of boundaries traversal is the following.

Entities of the grid surface dimension form intersections of lower dimensions than the dimension of the entity.
Therefore, each entity belongs only to one surface and can be traversed the following way.

\vspace{8pt}

\begin{listing}[!h]
\caption{The traversal of boundary entities of the grid surface dimension}
\begin{minted}{c++}
auto exec = [&](const auto orientation, const Coordinate& basis) {
  constexpr int orthogonalOrientation = EntityDimension - orientation;

  forBoundary(orthogonalOrientation, orientation, basis);
};

ForEachOrientation<Index, EntityDimension, Dimension>::exec(exec);
\end{minted}
\end{listing}

\vspace{8pt}

For all other entities, intersections are formed.
In addition to this, there are two other cases.
The first case is when the entity type has multiple orientations.
Then all entity orientations belong to the surface except the orientation which is orthogonal to the surface.
Hence, the orthogonal orientation must be omitted in the boundary entities traversal.
The second case is when the entity type has a single orientation, then all entities belong to the surface.
For both cases boundary intersection issue occurs, therefore the implementation remembers all traversed surfaces in the \texttt{isBoundaryTraversed} vector.
The implementation is presented in the Code Listing \ref{code:other-boundaries}.

\newpage
\begin{listing}[!ht]
\begin{minted}{c++}
auto exec = [&](const auto orthogonalOrientation) {
  auto exec = [&](const auto orientation, const Coordinate& basis) {
    forBoundary(orthogonalOrientation, orientation, basis);
  };

  ForEachOrientation<
    ..., isDirectedEntity ? orthogonalOrientation : -1
  >::exec(exec);

  isBoundaryTraversed[orthogonalOrientation] = 1;
};

DescendingFor<orientationsCount - 1>::exec(exec);
\end{minted}
\caption{The traversal of other boundary entities}
\label{code:other-boundaries}
\end{listing}

In both cases, the implementation calls \texttt{forBoundary} lambda functions, which start traversal of bottom and upper surface along with orthogonal orientation.
On lines 3-10 the program adjusts indices to not iterate intersection entities several times.
On lines 12, 15-16 start and end indices are adjusted to traverse only surface entities.

\begin{listing}[!h]
\caption{The traversal of boundary entities along orthogonal axis}
\begin{minted}{c++}
Coordinate start = from;
Coordinate end = to + basis;

if (EntityDimension != Dimension - 1) {
  #pragma unroll
  for (Index i = 0; i < Dimension; i++) {
    int offset = (!isDirectedEntity || basis[i]) && isBoundaryTraversed[i];

    start[i] = offset;
    end[i] = end[i] - offset;
  }
}

start[orthogonalOrientation] = end[orthogonalOrientation] - 1;

Templates::ParallelFor<Dimension, Device, Index>::exec(start, end, ...);

// Skip entities defined only once
if (!start[orthogonalOrientation] && end[orthogonalOrientation]) return;

start[orthogonalOrientation] = 0;
end[orthogonalOrientation] = 1;

Templates::ParallelFor<Dimension, Device, Index>::exec(start, end, ...);
\end{minted}
\end{listing}

\section{Unit testing}

After the implementation, several problems with unit testing of the solution arose.
Firstly, there are hundreds of edge cases for different grid dimensions and configurations.
Secondly, the implementation must be tested on multiple device architectures.
Thirdly, the solution actively uses template metaprogramming, therefore, multiple specializations of methods must be tested.
To address all these issues several test environments were written to test the implementation of the grids.
In general, the test environment must be additionally tested to ensure the validity of the test.

All tests were written using the modern C++ unit testing framework GTest \cite{GTest}.
GTest provides the solution to implement generic test cases by using \texttt{TYPED\_TEST} macro.
The whole documentation on the \texttt{TYPED\_TEST} can be found in \cite{GTest}.

\subsection{Test of grid properties}

The grid templates provide multiple getter and setter methods to configure the grid.
To test these methods, the test environment uses the following approach.
It configures the grid and instantiates a \texttt{TNL::Array} container.
The test environment traverses the container in parallel and on multiple devices.
During the traversal, the test environment gets the value from the grid using the index in the container and stores it in the container.
Finally, the test environment traverses the container once more, but the traversal is done sequentially and the values are verified.

\subsection{Test of traversal and grid entities}

To test traversal methods, the test environment uses the following approach.
It configures the grid and instantiates entity storage.
Then it calls the tested traverse method and stores all information about the entity in the entity storage.
Finally, it traverses all grid entities once more, but the traversal is done sequentially and the entity values are verified.
Additionally, the entities of the grid possess the following property: the set of all entities is formed by the union of interior and boundary entities.
Using this property the traversal methods were additionally tested.

\subsection{Test of neighbours}

To test the neighbour entity access, the test environment uses the following approach.
It configures the grid and instantiates entity storage.
Then it calls the tested traverse method with the body, inside which the grid accesses the specific grid entity neighbour.
The neighbour entity is stored in the entity storage.
Finally, the test environment traverses all grid entities once more, but the traversal is done sequentially and the entity values are verified.