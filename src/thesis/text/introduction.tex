
\chapter*{Introduction}\addcontentsline{toc}{chapter}{Introduction}\markboth{Introduction}{Introduction}

Grids (or meshes) are essential tools in computer science with a wide variety of applications.
All of them attempt to discretize a domain based on specific rules for calculations.
One of the notable members is the orthogonal structured grid.
Despite its simplicity, its implementation design has a crucial role in computations.
Poorly designed structured grids result in slow calculations or difficult problem declaration.
Grids are the main ingredient for the implementation of different numerical methods for solving partial differential equations (PDE).
And PDEs are ubiquitous in mathematically oriented fields, such as physics and chemistry.
They are widely used in fluid dynamics and other theoretical and practical fields.

Most numerical methods require performing thousands of iterations which makes them computationally expensive.
It must be noted, however, that some of them perform the same set of operations simultaneously for each entity in the grid, therefore, they are parallelizable.
GPUs are excellent for these tasks.

At the beginning of the thesis, the main goals will be outlined.

The theoretical part starts by describing the main components of the graphics processing unit (GPU) hardware architecture.
In addition to that, the thesis covers the basics of programming for CUDA in C++ and the profiling process of CUDA kernels.
After this, the thesis concentrates on the description of the Template Numerical Library (TNL).
CUDA programming model only provides developers with essential primitives needed to create applications for GPUs.
Therefore, different open-source libraries were developed to simplify programming for CUDA.
TNL is one such library.
The theoretical part ends with the description of the finite difference method.
At first, it defines the essential terms needed for understanding the numerical method.
Then it describes the process of the finite difference method and applies it to find the solution to the heat equation.

The practical part of the thesis begins with the description of the structured orthogonal grid implementation in the TNL.
At the end of the grid design description, the thesis covers the process of testing the implementation.
Then the thesis will describe the implementation of the convolution operator and the techniques used in its implementation on CUDA.
Afterwards, the thesis will describe the results of the improvements in the TNL library.
Finally, it will discuss the benchmark results of the heat equation numerical solver.
Additionally, the heat equation has a pseudo-analytical solution, which uses the convolution operator.
In the end, the thesis will show the similarities between the heat equation numerical solution and the heat equation pseudo-analytical solution.
